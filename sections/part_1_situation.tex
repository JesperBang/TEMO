
% How to comment 
% 2-3 pages !
% 1.How would you characterise the company based on the presentation?
% 2.What is/are the key problem(s)? (be aware that the key problem(s) is not necessarily identical with the challenge and might only be implicitly given in the challenge).
% 3.What is the consequence of the present situation?
% 4.What impact does the situation have on the three organisational levels?
% 5.Who is/are the problem owner(s)?
% Use models and theories to describe the context of the company, strategy, organisational design, business models etc
% Combine these models and theories with the data presented in the company challenge.


Worklife Barometer ApS is  Danish Startup Established in 2014, born with the purpose of preventing sick leaves,who is adopting a validated questionnaire for the identification of wellbeing. Its strategy is based on interposing itself between the companies and their employees so to guarantee a more fluid dialog on personal matter.     
In order to reach this achievement, it proposes three kind of products regarding different wellbeing aspects: Howdy wellbeing, Howdy body and Howdy feedback.\cite{howdywebsite}

\noindent Howdy Wellbeing,[2] is based on a questioner of 5 questions, asked to the employees 2 times a month, where is monitored their mental wellbeing. 
Furthermore, It gives, in real time, an overview of their status, underlining the first symptoms of sickness.
When a sign of stress is detected, Howdy proceeds by making a psychologist calling the stress employee in order to guide and assistance. With the purpose of avoiding the employee to develop further symptoms.

\noindent On the other hand, to ensure a completed body screen, Howdy Body \cite{howdywebsite} monitors also the physical pain and provides physiotherapist’ consultant to guide the employees through a healing process, offering exercises to correct and fix the pain source.

\noindent Often companies do not have the right recourses, or they do not the better approach to organize annual employees’ surveys. For this reason, Howdy offers Howdy feedback,[4] to give the chance to collect employees’ opinions anytime it is felt necessary by the company.

\noindent Currently, its team is composed of more than10 employees skilled in IT, sales, marketing and product development and supported by external professionals (psychiatrists and physiotherapists). Worklife Barometer ApS is boasting 14000 users spared on approx. 70 companies, (such as Simens, Skat, Scania, Orsted and Københavns commune) between Denmark, Sweden and the United Kingdom. Its goal is to reach new customers and to expand in new counties in the next years.\cite{howdywebsite}

\noindent Thanks to its activity, it helped to prevent 1.6\% sick leaves and 12\% of the users claimed that it has led them to talk to their manager about wellbeing-problems. \cite{howdywebsite}  

\noindent Even though Howdy is gathering positive feedback from its customers, it observed some problems where they could not find an explanation. So, in the following report, Howdy will be analyzed in order to detect the reasons why:

•	customers are not onboarding Howdy to get a greater extend
•	the pension companies are reluctant to rollout Howdy in big scale

\noindent In particular it has been asked to focus on GTM (go-to-make) strategy, especially on pricing strategy, and to identify the right team (in terms of roles) that will bring the company to a further step.\cite[s.39]{oneofthepresentations}

\noindent In this report, business models will be adopted with the aim to look at Howdy in different prospective so to draw a wider and completed picture of the company.  Thereby, there will be an inspection through all the company´s levels in order to point out at the possible improvement solutions. This study will start with the identification of key problem and to problem owner, which are:


\begin{itemize}
  \item key problem : HOWDY is not successfully getting new clients and scaling the business\cite[s.38]{oneofthepresentations}
  \item problem owner: Rasmus Hartung (CEO)\cite[s.5]{oneofthepresentations}
\end{itemize}





\subsection{Assumptions - Kazi}

In our analysis and when building up our solution, we made a couple of assumption:

\begin{enumerate}
    \item Howdy has a bonus plan for the sales department and it is sufficient.
    \item Howdy has the resources to hire new people 
    \item Howdy has not being marketing sufficiently 
\end{enumerate}

\noindent Howdy has a bonus plan for the future sells team so we believe it would be better to propose them a bonus for sell they make which will motivate the sells team to sell Howdy’s product in a greater extent.

\noindent Howdy’s marketing focused on end corporations or pension companies both but pension companies run their own campaign where Howdy is not allowed to participate. So they are pretty much dependent on pension companies which is not enough to get more customer onboard. We acknowledge a much larger campaign and negotiation with pension companies with promoting Howdy’s product can develop the marketing strategy.