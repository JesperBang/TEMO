
% How to comment 
% 2-3 pages !
% 1.How would you characterise the company based on the presentation?
% 2.What is/are the key problem(s)? (be aware that the key problem(s) is not necessarily identical with the challenge and might only be implicitly given in the challenge).
% 3.What is the consequence of the present situation?
% 4.What impact does the situation have on the three organisational levels?
% 5.Who is/are the problem owner(s)?
% Use models and theories to describe the context of the company, strategy, organisational design, business models etc
% Combine these models and theories with the data presented in the company challenge.


Worklife Barometer ApS is a Danish Startup Established in 2014, born with the purpose of preventing sick leaves, adopting a validated questionnaire for the identification of wellbeing. Its strategy is based on interposing itself between the companies and their employees so to guarantee a more fluid dialog on personal matter.     
In order to reach this achievement, it proposes three kind of products regarding different wellbeing aspects: Howdy body, Howdy wellbeing and Howdy feedback. 
 
Currently, its team is composed of more than 10 employees skilled in IT, sales, marketing and product development and supported by external professionals (psychiatrists and physiotherapists).

Worklife Barometer ApS is boasting 14000 users spared on approx. 70 companies, (such as Simens, Skat, Scania, Orsted and Risultati relativi a Københavns commune) between Denmark, Sweden and the United Kingdom. Its goal is to reach new customers and to expand in new counties in the next years.

Thanks to its activity, it helped to prevent 1,6\% sick leaves and 12\% of the users claimed that it has led them to talk to their manager about wellbeing-problems.   

\noindent From our analysis we found that... :
\begin{itemize}
    \item \textit{Key problem:} HOWDY is not successfully getting new clients and scaling the business.
    \item \textit{Problem owner:} Rasmus Hartung (CEO).
\end{itemize}



\subsection{Assumptions?}

\begin{enumerate}
    \item The reward system for the sales department is not suff.
    \item Howdy has the resources to hire new people 
\end{enumerate}


