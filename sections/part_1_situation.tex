
% How to comment 
% 2-3 pages !
% 1.How would you characterise the company based on the presentation?
% 2.What is/are the key problem(s)? (be aware that the key problem(s) is not necessarily identical with the challenge and might only be implicitly given in the challenge).
% 3.What is the consequence of the present situation?
% 4.What impact does the situation have on the three organisational levels?
% 5.Who is/are the problem owner(s)?
% Use models and theories to describe the context of the company, strategy, organisational design, business models etc
% Combine these models and theories with the data presented in the company challenge.


Worklife Barometer ApS is a Danish Startup Established in 2014, born with the purpose of preventing sick leaves, by adopting a validated questionnaire for the identification of wellbeing. Its strategy is based on interposing itself between the companies and their employees to guarantee a more fluid dialog on personal matters.     
In order to reach this achievement, it proposes three kinds of products regarding different wellbeing aspects: Howdy wellbeing, Howdy body and Howdy feedback.\cite{howdywebsite}

\noindent Howdy Wellbeing is based on a schema of 5 questions, asked to the employees 2 times a month.
Furthermore, it gives, in real time, an overview of their status, underlining the first symptoms of sickness.
When a sign of stress is detected, Howdy proceeds by making a psychologist call the stressed employee to offer assistance and to guide them. This is done with the purpose of avoiding a worsening of the situation.

\noindent Often companies do not have the right resources, or they do not know how to organize annual surveys for their employees. For this reason, Howdy offers Howdy feedback,[4] to collect employees’ opinions anytime the company deems it necessary.

\noindent Currently, Howdy is composed of more than 10 employees skilled in IT, sales, marketing and product development and supported by external professionals (psychologists and physiotherapists). Worklife Barometer ApS is boasting 14000 users spared on approx. 70 companies, (such as Simens, Skat, Scania, Orsted and Københavns commune) between Denmark, Sweden and the United Kingdom. Their goal is to reach new customers and to expand to other countries in the next years.\cite{howdywebsite}. Thanks to Howdys product and service, 1.6\% of sick leaves have been prevented and 12\% of the users claimed that it has led them to talk to their manager about wellbeing-problems. \cite{howdywebsite}  

\noindent Even though Howdy is gathering positive feedback from its customers, it has been observed that they have not managed to on-board enough customers and are currently losing money. In the following report, Howdy will be analyzed in order to detect the reasons why.

\noindent In particular it has been asked to focus on GTM (go-to-make) strategy, especially on pricing strategy, and to identify the right team (in terms of roles) that will bring the company to a further step.\cite[s.39]{oneofthepresentations}

\noindent In this report, business models will be used in an attempt to look at Howdy from different perspectives. This should draw a wider and more clear picture of the company. Thereby, it will be possible to inspect the company at all levels and to point out possible improvements. This study will start with the identification of the key problem and a problem owner, which are:


\begin{itemize}
  \item key problem : HOWDY is not successfully getting new clients and scaling the business\cite[s.38]{oneofthepresentations}
  \item problem owner: Rasmus Hartung (CEO)\cite[s.5]{oneofthepresentations}
\end{itemize}





\subsection{Assumptions}

While doing the analysis and building up the solution, a couple of assumptions were made. These are:

\begin{enumerate}
    \item Howdy have the resources to hire new people 
    \item The bonus plan for the sales department is adequate 
    \item Howdy has not been marketing sufficiently 
\end{enumerate}

\noindent As Howdy has mentioned plans of expanding the sales team they should have resources to hire new people.

\noindent Howdy has a bonus plan for the sales team that we believe motivates the sales team to sell Howdy’s product and the teams performance should therefore be adequate.

\noindent Howdys marketing is focused on pushing their product through partners like the pension companies but these companies only run campaigns for their own product and Howdy is not allowed to participate. This means that they pretty much are dependent on their partners who do not increase on-boarding of customers at a particularly fast rate. We acknowledge that a much larger campaign and negotiation with pension companies regarding the promotion of Howdy’s product can develop the marketing strategy.
