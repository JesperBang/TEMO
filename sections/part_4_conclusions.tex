% 3-4 pages!
% • Present your recommendations based on your analysis
% • What would you recommend the company and the relevant actors to do?
% • How can your recommendations contribute to meet the challenge and give an edge to the competitive advantage of the company
% • Reflect on how your solution/recommendations can be implemented in the company
% • Describe how the key issues from your analysis and your recommendations are likely to affect the company from a strategic, tactical and operational level using your knowledge from the models and theories.
% • Identify and outline the possible extra information you need in order to implement your recommendations.
Our recommendations will now be presented from the implementation point of view, either long-term or short-term. This is based on how important the solution is, how easy it is to implement and how many problems it can possibly address.



\subsection{Short-term : Marketing + pricing}
Due to the nature of the situation explored in this report, our first and most important recommendation is to focus on marketing. This solution does not demand any organisational restructuring and thus can be implemented promptly.

\noindent By focusing on marketing, Howdy can attract new customers and increase the profitability of the company. Moreover, it was identified that relying on the health care providers for advertisement has not been efficient and has only served to decrease their profit margins. Howdy has the backing and the statistics from their current customers available, which can be used to advertise and push their product. Marketing was identified as a opportunity in the SWOT analysis and should be implemented at the strategic level by the sales director.

\noindent The second short-term solution that is recommended is to improve the way the pricing system is organised. Similarly to the previously discussed solution, it does not require any organizational restructuring and can be implemented along the other solutions. The solution that we described does not change the cost for the current clients, it just aims to clear up any confusion and make the future clients more secure in their purchase and what exactly they are paying for. This can be easily implemented at the tactical level of Howdy. 


\subsection{Long-term : Organisational restructuring}

The previous section has presented what aspects Howdy could work on in the short-term. Here we will be discussing solutions regarding Howdys' organizational structure problems, including what departments that should be expanded and where in-sourcing should be considered. 

\noindent The developing department is continuously interrupted, and is not able to execute strategy meetings, since the maintenance department is regularly asking for support. Moreover, the cost of external psychological support is cutting profit margins while not helping the company in gaining more customers.

%\noindent Assuming that Howdy has the resources to hire new personnel, it should consider enforcing the departments where more resources are needed.

\noindent First of all, by hiring new staff in the maintenance department, it is possible to avoid possible interference with the devolving department. Before implementing this solution Howdy should first asses the maintenance workload as hiring unnecessary people would only increase the companies cost without any 


\noindent Secondly, during one of the CEO´s interviews, he declared that it would be cheaper to insource the psychological support, but some considerations have to be mentioned first. Having an external support team offers a lot of flexibility, especially when the demand is not high, but since Howdy is aiming to scale up their business and on-board more customers, in the long run it will be a good idea to invest in having their own psychologists. 
 

% \noindent\textbf{No recourses to hire} $|$ In this case that our assumptions are no longer true, so the company does not have the resources to hire people, temporary solutions can be adopted. First, some of the developing employees could be transferred in the maintenance and just the needed will be kept in the department. Furthermore, when the workload is low, the skilled employees, previously transferred to the maintenance department, could help in the developing department. This kind of solution can take place just for a short period since the developing department is the key to being competitive in the market, so essential and unreplaceable. In the matter of insourcing the psychologist support, it is a department that Howdy must own. Once the first psychologist is hired, and she/he will take some of the workloads that before was given to the external support and the results will be seen soon, without further investments. So, it is an investment that founds itself.


\subsection{Conclusion}
After using several models to analyse Howdy as a company, their business and the market that they operate in, the key problem and multiple sub problems have been identified. Using the root cause analysis relations between the problems were presented and the causes of the problems found. Throughout part 3 different solutions to each of the causes were presented and advantages and disadvantages of each were discussed. In the end short-term and long-term changes were suggested for the company. In the short-term they should rework their pricing model to make it easier on the customer and shift the focus of their channels to market and sell directly to customers themselves instead of through partners. In the long-term Howdy could have a focus on expanding the sales team to increase sales, expanding the maintenance team to keep up with the workload and hiring psychologists when needed instead of outsourcing the work.

\subsection{Robustness check}

To analyze our assumption in what extend our solution and recommendation will change we did the following robustness check:

\noindent\textbf{Hiring new employees} $|$ Howdy has the resources to hire employees based on their needs. They can expand their sales team to boost-up their sales as well as marketing department who will analyze business cases in deeper consent to established marketing strategy that will bring success in the company. They can also add member in their maintenance team so that their day to day task do not get interrupted.

\noindent \textbf{Howdy cannot manage financial resources to hire employees} $|$ If expanding is not possible Howdy can made an organizational restructure. This may slow down the development, but the constant activity of the organization is progressively significant.

\noindent \textbf{Bonus scheme} $|$ Howdy can implement bonus scheme to their future sells department which can motivate their employees in a greater extend to bring a fruitful outcome of their work. If bonus scheme is not adequate, they will need to add a better reward system to their short-term solution. Since it does not require organizational changes it will be easy to implement.

\noindent \textbf{Howdy has been marketing sufficiently} $|$ Though Howdy is marketing their product through some channels but we believe Howdy has been going through the wrong channels so our solution still stands which does not require any long time strategy and can be implemented instantly. They can develop their marketing strategy by doing stuff like social media coverage, email marketing, introducing free workshops and webinar
